%%%%%%%%%%%%%%%%%%% Section 8 %%%%%%%%%%%%%%%%%%%%%%%
\vspace{-1ex}
\section{Preliminary}
\label{sec-pre}

\vspace{-1ex}
We start with the notions of graphs.

\stitle{Graphs}. We consider an 
attributed graph
$G$ = $(V,E,F_A)$ with a finite set of
nodes $V$, and
a finite set of edges $E\subseteq V\times V$.
Each node $v\in V$ has a
{\em node tuple} $F_A(v)$ =
$\{(A_1, a_1), \ldots, (A_n,a_n)\}$
defined on a set of node attributes $\A$,
where a pair $(A_i, a_i)\in F_A(v)$
states that the attribute $v.A_i\in\A$ has
a value $a_i\in\adom(A_i)$. Here
(a) $\A$ refers to a set of all the node
attributes seen in $G$; and (b)
$\adom(A_i)$ is a finite {\em active domain} of
attribute $A_i$ in $G$, and contains
all the values of $v.A_i$, where $v$ ranges
over all the nodes in $V$.





